Es recomendable que el guía de cada grupo prepare su propia motivación a escuchar el Evangelio, pero en caso de no hacerlo puede leer una preparada. Si decide prepararla, es importante recalcar que la motivación no debe ser demasiado larga, ni muy explicativa. La idea es que una vez hecha la motivación los que participan de la liturgia quieran escuchar la lectura. Es como si estuviesen invitándolos a comer un plato muy rico, no les describirían los ingredientes del plato, sino que darían razones para comerlo.

\hfill \\ \hfill
''
\hfill \\ \hfill

Se puede dejar un rato de silencio antes de leer el Evangelio para entrar con la mente limpia y sin preocupaciones.

\newpage
